\section{Related work}

Different attempts at the security analysis of Garmin software are available online.
There were attempts to ethically hack a Garmin watch\cite{kth-ethical-hacking,kth-audit}, analyzing different vectors of attack such as the Garmin Connect website, watch communication, and third-party apps.
However, they did not go deep into any of those topics.

Others tried to reverse engineer Garmin’s Virtual Machine and do a security analysis of the environment\cite{broken-vm,compromising-garmin-watches}.
They were able to get access to the firmware of the watch.
With the help of unofficial documentation of the firmware format\cite{firmware-format}, they proceeded to reverse engineer the bytecode.
During the analysis, they determined which parts of the code are responsible for the third-party apps' execution.
Thereafter, they found multiple vulnerabilities concerning the app runtime that have been disclosed to Garmin.
Some of them affected over a hundred devices for around 7 years.

While they explored the specific topic deeper, there is still a long list of things that could be analyzed in more detail.
Previous attempts used an available beta firmware for the analysis and used watches released over 4 years ago.
Since then, many new models have been introduced as well as new firmware versions with new features and expanded developer SDK.

Considering the size of Garmin's ecosystem and the number of different components, the existing, publicly available analysis is far from being comprehensive.


