\section{Introduction}


\subsection{Garmin fitness watches}

In the past years fitness trackers experienced a high growth in popularity.
As the technology develops, the devices get more affordable and are able to track more metrics.
While it improves the insights into health and fitness tracking, it also means that it collects more sensitive data about the users.

Garmin fitness watches can accompany the users day and night, being able to last even over a week on a single charge.
The company has a wide range of options, having released over 100 different watches with third party apps support.
The devices are equipped with a large number of sensors such as heart rate, pulse oximeter, temperature, gps, compass, accelerometer, gyroscope, altimeter, barometer, light sensor.
This allows collecting and calculating an enormous amount of different sensitive metrics.
As a result, the potential consequences of unauthorized access to the user's data are severe.

The watches can work independently, collecting and processing the data offline.
However, Garmin offers an extensive set of software to improve the user's experience.
There are multiple apps available for iPhone and Android phones to manage the watch and the data.
This means that the data is being sent over a Bluetooth protocol from the watch to the phone.
Additionally, Garmin encourages users to synchronize the data with their cloud.

In case the provided software does not fulfill users' requirements, it is also possible to install third party apps on the watch.
Garmin maintains an ecosystem of tools for the developers to create apps and publish them in a Garmin store.
It makes the watch extensible and more universal, but it also allows for a third party developer to gain access to some of the user data.

The described software offers a powerful toolset for the users to analyse their fitness data.
However, it also creates a large number of possible vectors of attack.
It is also important to notice that the Garmin software is not open source, which makes it more difficult to analyse the security of the software.
Garmin provides a contact form on their website to report security vulnerabilities.
Unfortunately, they do not offer a bounty program.

In this report, a broad analysis of different attack vectors has been performed.

Overall, during the analysis, the following privacy and security concerns were found:
\begin{itemize}
    \item Garmin uses deprecated SHA-1 for app signing
    \item Android apps installed on the phone, without any permissions can detect that the user uses Garmin devices and the type of the device
    \item Communication between the third party apps and their companion apps are in plain text, available to all apps installed on the Android phone.
    I created a proof of concept app that can eavesdrop on the messages sent to the phone.
    \item Permission system of the third party watch apps is not as well separated into different categories as it is in Android, iOS phones.
    \item Fuzzy testing suggests that it may be possible to escape the virtual machine running the third party apps.
\end{itemize}


\subsection{Software description}
Garmin provides multiple mobile apps to manage the watch and the data.
\begin{itemize}
    \item Garmin Connect — the main app for managing the watch and the data.
    \item Garmin Explore — app for managing the watch and the data, focused on the offline usage.
    \item Garmin Connect IQ Store — app store for third party apps, that can be installed on the watch.
\end{itemize}

If the user decides to synchronize the data with the cloud, it is sent to the Garmin Connect services.
Then it can be accessed through the website or the mobile app.

Additionally, Garmin has developed a proprietary programming language, Monkey C running in a virtual machine for creating third-party apps.
The developers are provided with an SDK, containing all the necessary tools for creating and testing the apps.
Monkey C language has an integration with Visual Studio Code through the extension.
The apps can be tested either on a real device or on a simulator, provided with the SDK\@.
There is an official documentation describing the development process.

After a preliminary analysis, the following scripts and tools were found in the SDK:

\begin{itemize}
    \item monkeybrains.jar - java archive, compiles and builds the project
    \item simulator - binary ELF file, simulates a chosen Garmin device
    \item monkeydo - executes Connect IQ executable on the simulator
    \item Other scripts: barrelbuild, barreltest, connectiq, era, mdd, monkeyc, monkeygraph, monkeydoc, shell, monkeymotion
\end{itemize}

\subsection{Testing environment}
For the analysis the decision was made to use Linux operating systems as it is a popular choice in the system security community and a large number of materials use it.
The chosen distribution was Kali Linux because it comes with already preinstalled toolset.

The testing is executed on the watch Garmin Fenix 7S Pro, released in May 2023.
The third party applications are investigated when using Connect IQ 6.3 SDK, released in August 2023.

Garmin supports both Android and iOS phones for managing the watch.
I focused on the Android version of the software.
Android is more open and easier to analyse, and I did not have access to Apple devices.


For the purpose of the analysis, when Garmin watches are mentioned, this report focuses on Garmin Fenix watches series 7.
Other devices may offer different hardware and software configurations.



